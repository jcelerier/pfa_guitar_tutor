\chapter{Player}

\indent Pour la partie Player du logiciel, celle qui permet à un étudiant de joueur un morceau de guitar, le travail a été centré sur l'ajout de deux nouvelles fonctionnalités
qui étaient vu comme les plus importantes :\\
\begin{itemize}
\item Une installation facile et multi plateformes (cf partie installation).
\item Une interface permettant le choix des chansons.\\
\end{itemize}
\indent Ce sont ces deux points que nous allons développer maintenant.\\

\section{Choix des chansons}
\indent Avec la création de la partie éditeur, la bibliothèque des musiques jouables avec le logiciel a été étendue.
Il existe désormais un répertoire comprenant pour chaque morceau jouable un fichier .txt et un .wav, ces deux fichiers contiennent toutes les informations nécessaire pour que le morceau soit jouable dans le Player de Guitar tutor. 

\indent Pour que le morceau soit joué, il faut le sélectionner au préalable. Pour ce faire, nous avons introduit une ComboBox, afin de lister facilement les choix possibles. 
Il suffit alors d'indiquer à partir de quel répertoire nous pouvons choisir une chanson, et le Player présente la liste des musiques disponibles dans ce répertoire. 

\indent Le joueur n'a plus qu'à selectionner un titre, une fois celui-ci sélectionné, la grille d'accords correspondant à la chanson est affichée, et sa lecture peut commencer.


%%% je sais pas quoi dire d'autre... Moi non plus ... %%%
