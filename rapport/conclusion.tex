\section*{Conclusion}

\subsection*{Un objectif atteint}

Nos clients avaient été très clair à ce sujet lors de la présentation du projet: l'objectif était, à la fin du PFA, de fournir deux programmes totalement fonctionnels, et prêts pour une démonstration en école de musique.

Nous avons remis à nos clients, peu de temps avant la date finale des projets, une version en \textit{pre-release} de GuitarTutor afin qu'ils puissent avoir un réel aperçu du programme, et que nous puissions, le cas échéant, corriger quelques détails. Ceux-ci semblent avoir été particulièrement satisfaits du travail que nous avons réalisé, et c'est là le plus important pour notre propre satisfaction. Le deuxième point est sans doute le fait que nous pensons clairement avoir atteints les objectifs qui étaient demandés. Nous avons en effet réalisé une suite logicielle qui, à notre sens, semble parfaitement utilisable dans le contexte cible - c'est-à-dire en école de musique -, et c'était d'ailleurs ce vers quoi nous avons orienté notre développement. La compatibilité du projet sur les systèmes Windows et Mac est elle aussi parfaitement respectée, malgré les difficultés loin d'être négligeables auxquelles nous avons été confrontés.

En sus, la reprise de l'ancien code nous a fait comprendre l'importance de produire un code de qualité et de tout faire pour faciliter la reprise ultérieure, éventuellement par des personnes tierces. C'est pourquoi nous avons soigné notre code source en fournissant une documentation exhaustive de notre travail ainsi qu'un code propre et clair (du moins, selon nos critères). Des optimisations ont également été apportées alors qu'elles n'étaient pas explicitement demandées, telles que le nettoyage des librairies inutilisées, la gestion du format MP3, la limitation des fuites mémoires, ou encore l'utilisation d'OpenGL pour améliorer les performances.

\subsection*{Un travail d'équipe, une expérience de travail profesionnel}

Le PFA aura également été l'opportunité d'apprendre à gérer un véritable travail en équipe, d'adopter une méthode de développement, ainsi que de connaître des outils comme Git ou la librairie Qt, qui, sans être incontournables, font de même bonne figure sur un CV. C'était aussi l'occasion de débattre et de rechercher des informations sur la manière d'implémenter les choses; ou encore de travailler à partir d'un code existant.

En clair, il s'agissait de savoir se comporter dans un contexte quasi-professionnel, et c'est sans aucun doute une expérience extrêmement enrichissante.
%Ce que nous a apporté le projet
%Apprentissage de librairies complexes
%Si chacun veut mettre un mot personnel ici...
