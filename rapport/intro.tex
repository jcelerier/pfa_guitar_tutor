\section*{Introduction}
%Dans cette partie, j'ai repris quasiment mot pour mot l'intro que j'avais écrite pour le cahier des charges.

\subsection*{Serious game}

En quelques années, les jeux vidéos ont su attirer un public immense et éclectique. Alors qu'ils étaient réservés, il y a encore quelques années, à un public bien ciblé, les éditeurs ont su s'ouvrir à toute une gamme de joueurs jusqu'alors insoupçonnés - l'appellation casual gamer était née. Aujourd'hui, les recherches dans le domaine vidéo-ludique vont encore plus loin. A l'aide de technologies toujours plus poussées, toujours plus proches du joueur et de son environnement, il est désormais possible d'interagir avec lui, grâce à ses mouvements par exemple. Et pourquoi pas par le son? Les jeux de rythme, remontant pourtant aux bornes d'arcade des années 1990, ont connu un immense succès avec la sortie de jeux tels que Guitar Hero en 2005, qui a ouvert la voie à de nombreux remakes que sont Just Dance ou encore Band Hero. En parallèle, les jeux à caractère éducatif, ont été pendant longtemps recalés au rang de “sous-jeux” de par leurs gameplays souvent repoussants et des graphismes généralement peu soignés -
évidemment, les budgets ne sont pas forcément les mêmes.

Le projet GuitarTutor se place au croisement de ces deux genres pour s'inscrire dans celui très fermé des serious games: pourquoi ne pas apprendre en s'amusant? (En l'occurrence l'apprentissage de la guitare). Ce jeu s'adresse à des élèves d'écoles de musique débutant la pratique de l'instrument. L'objectif est de les encourager à jouer et à s'entraîner en dehors des heures de cours avec leurs professeurs, en faisant intervenir cet univers ludique, pour les inciter à persévérer et à travailler d'eux-mêmes.

\subsection*{Le logiciel avant le PFA}

Le logiciel GuitarTutor reposait sur des travaux de recherches du \ac{LABRI}, ainsi que sur le travail fourni par des élèves de l'ENSEIRB-MATMECA dans le cadre des Projets au Fil de l'Année (PFA) de 2011-2012. La base du projet consistait en l'analyse d'accords en temps réel, c'est-à-dire pouvoir donner précisément le nom de l'accord qui est donné en entrée audio de l'ordinateur. La librairie EHPCP a donc été codée à cet effet. En complément, le développement d'une interface graphique, à l'aide de la librairie Qt, a été réalisée. Celle-ci assurait le fonctionnement basique d'un logiciel permettant de visualiser une liste défilante d'accords ainsi que le résultat de la partie analyse de l'entrée audio, en comparant cette donnée au résultat attendu.

Afin de faciliter l'écriture des fichiers partitions, un second logiciel avait été mis en place à destination du professeur. Il permettait une édition totalement manuelle de grilles d'accords, ainsi qu'une édition assistée. Ce second mode d'édition demandait à l'utilisateur de marquer les accords d'un morceau joué simultanément par l'appui sur une touche de son clavier. Dans un deuxième temps, il suffisait d'indiquer quels étaient les accords qui correspondaient à chaque pulsation.

\subsection*{Notre objectif}

L'objectif de notre projet était donc de rendre le logiciel existant accessible. Son interface d'origine avait, en effet, plus la carrure d'un prototype que celle d'un produit fini. Un soin particulier devait être apporté à l'éditeur de partitions afin d'optimiser au mieux l'expérience utilisateur. Selon les demandes clients, GuitarTutor devait être un logiciel fini et livrable pour avril 2013.
