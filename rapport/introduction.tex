\chapter{Introduction}
Ce projet proposé par les chercheurs du Labri, consiste à réaliser un logiciel qui sert de un support pédagogique à une école de musique afin d'aider les élèves à jouer de la guitare. En effet, il s'agit d'un jeu interactif permettant aux élèves de jouer des morceaux de musiques édités préalablement par les professeurs grâce à une interface faisant défiler les accords à jouer et d'analyser par la suite leur performance (élèves). Nous avons commencé l'implémentation en se basant sur le logiciel exisant \textbf{I-Guitar} qui a été développé par les chercheurs.

\section{Rappel des fonctionnalités}
Rappelons les fonctionnalités à développer pour la réalisation d'un logiciel qui correspond pleinement aux attentes des clients.\\

\textbf{Player}
\begin{itemize}
\item Installation facile et multi-platforme
\item Interface intuitive et facile à utiliser
\item Choix de la musique à jouer
\item Retour sur la performance de l'utilisateur
\end{itemize}

\textbf{Editeur}
\begin{itemize}
\item Création des tablatures 
\item Edition des tablatures
\item Numérisation d'un morceau
\end{itemize}
