\chapter{Introduction}
Ce projet proposé par les chercheurs du Labri Mr. Pierre Hanna ainsi que Mr. Matthias Robine, consiste à réaliser un logiciel qui sert comme un support pédagogique à l'école de musique Rock et Chanson afin d'aider les élèves à jouer de la guitare. En effet, il s'agit d'un jeu interactif permettant aux élèves de jouer des morceaux de musiques édités préalablement par les professeurs grâce à une interface faisant défiler les accords à jouer et d'analyser par la suite leur performance (élèves). Nous avons commencé l'implémentation en se basant sur un projet du Labri nommé \textbf{GuitarTutor}.

\section{Rappel des fonctionnalités}
Rappelons les fonctionnalités à développer pour la réalisation d'un logiciel qui correspond pleinement aux attentes des clients.\\

\subsection{Player}
La partie player est dédiée à l’apprentissage de la guitare, grâce à une interface de jeu faisant défiler les accords à jouer avec la musique. Le joueur doit
avoir un retour de sa performance, dans le but d’analyser et d’améliorer celle-ci. Le \textbf{player} assure les fonctionnalités suivantes :
\begin{itemize}
\item Installation facile et multi-platforme
\item Interface intuitive et facile à utiliser
\item Choix de la musique à jouer
\item Retour sur la performance de l'utilisateur
\end{itemize}

\subsection{Editeur}
L’édition des tablatures faisait partie des besoins exprimés par les professeurs de l’école de musique, c'est pour cela que nous avons opté, lors de la spécification des besoins, pour la création d’un nouveau logiciel permettant d'éditer des tablatures. L'\textbf{éditeur} regroupe les fonctionnalités suivantes :
\begin{itemize}
\item Création des tablatures 
\item Edition des tablatures
\item Numérisation d'un morceau
\end{itemize}

Nous rappelons que l'installation doit pouvoir se faire sur une distribution Mac et Ubuntu, de la manière la plus simple (double-clic).
