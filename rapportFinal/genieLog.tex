\chapter{Partie Génie Logiciel}
\section{Méthodologie}

Pour la reprise et le développement du logiciel Guitar Tutor, le choix adopté a été de suivre une méthode AGILE. Nous avions opté pour la méthode SCRUM, qui consiste à faire des sprints de développement, et des réunions régulièrement de mise au point après l’intégration des modules en fin de sprint. Le but étant que le logiciel puisse tourner à la fin de chacun des sprints et que chaque développeur soit occupé pour chacun de ces sprints. Lors de ces réunions, nous vérifions que le code intégré est bien fonctionnel.%%, et que le logiciel puisse tourner.\\
Nous avions planifié 5 sprints pour répartir les taches entre plusieurs binômes de travail, tout le long du temps de développement.  
Au fur et à mesure nous avons dû réduire le nombre de fonctionnalités, en privilégiant les plus essentielles, en accord avec nos clients sur l'ordre de priorité. En effet, le développement du programme s'est avéré bien plus difficile que prévu, avec du recul, notre cahier des charges était trop ambitieux.  Nous avons donc développé en priorité les fonctionnalités prévu pour l’éditeur, la possibilité de choisir un morceau dans le Player, et enfin une installation facile et rapide.

\section{Outil de suivi}

Pour le suivi notre projet, nous avons décidé d’utiliser la forge de l’ENSEIRB-MATMECA. Elle dispose en plus du dépôt subversion de différents outils de gestion. Nous avons notamment utilisé le diagramme de Gantt sur lequel nous avions créé les taches et les sous-taches à réaliser. Un système de filtre permet d'y voir rapidement les taches qui sont en cours de développement, terminées, et celles qui sont attribuées ou non attribuées.

\section{Analyse} 

L’expérience de la méthode agile n’a pas été concluante pour notre groupe de travail. Si elle est efficace dans le monde professionnel, nous avons éprouvé quelques difficultés à les mettre en œuvre, surtout dans notre cadre scolaire. En effet notre groupe de PFA est composé d’élèves qui suivent différentes options, et n’ont pas cours en même temps. Et si au début du semestre des créneaux étaient réservés au PFA, ils ont souvent été utilisé par l’administration comme créneau de rattrapage de cours. Sur la fin, nous n’avions plus du tout de créneau réservé à la réalisation du PFA, les réunions étaient donc difficiles à mettre en place. 
Nous avions prévu de faire des sprints de 2 semaines, mais nous nous sommes rendu compte que cette durée était trop courte, nous avons donc repoussé d’une semaine la fin des sprints. 
De plus, le temps passé à mettre à jour le diagramme de Gantt d’un sprint à un autre était considérable (report des taches non accomplies, attribution de toutes les taches à un binôme). Autant de temps qui finalement n’était pas utilisé pour le développement. 

\section{Reprise de code}

Nous avons sous estimé la charge de travail inhérente à la reprise du code originel du logiciel, et au temps de formation sur les librairies (FMODEX, Qt) et sur le développement sous mac. 
L’absence de documentation, de \textbf{readme}, ou d’autres aides à la compréhension du code nous a fait perdre du temps dans la maîtrise de celui-ci et à tâtonner avant de pouvoir commencer le développement.
De plus, nous avons dû nous familiariser avec le modèle MVC, utilisé dans le code fourni, avant de pouvoir interragir proprement.

\section{Documentation et tests}

Dans un but de pérennisation du code du projet, nous avons particulièrement fait attention à la rédaction des manuels. 
Nous avons rédigé un manuel de maintenance qui explique l’architecture du code ainsi que les points délicats de celui-ci, et un manuel d’utilisation à destination des clients qui explique comment installer et utiliser toutes les fonctionnalités des produits. (Vous trouverez ces manuels en annexes.)
Nous avions prévu de faire des tests avec l’école de musique afin prendre en compte leur dernière remarque sur les livrables. Malheureusement nous n’avons pas eu le temps de mettre en place cette réunion car nous n’avions pas fini le développement à temps. Nous avons donc exécuté ces tests nous même en essayant d’être le plus critique possible.
